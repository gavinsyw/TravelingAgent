\documentclass[10pt]{article}

\usepackage[a4paper, left=2cm, right=2cm]{geometry} % A4 paper size and thin margins

\usepackage{xcolor} % Required for specifying custom colours
\definecolor{grey}{rgb}{0.9,0.9,0.9} % Colour of the box surrounding the title

\usepackage{graphicx}
\usepackage[colorlinks=true, allcolors=black]{hyperref}
\usepackage{amsmath}
\usepackage{indentfirst}
\setlength{\parindent}{2em}

\usepackage[utf8]{inputenc} % Required for inputting international characters
\usepackage[T1]{fontenc} % Output font encoding for international characters
\usepackage[sfdefault]{ClearSans} % Use the Clear Sans font (sans serif)
%\usepackage{XCharter} % Use the XCharter font (serif)

\newcommand{\tabincell}[2]{\begin{tabular}{@{}#1@{}}#2\end{tabular}} 	

\begin{document}
%----------------------------------------------------------------------------------------
%	TITLE PAGE
%----------------------------------------------------------------------------------------

\begin{titlepage} % Suppresses displaying the page number on the title page and the subsequent page counts as page 1
	
	%------------------------------------------------
	%	Grey title box
	%------------------------------------------------
	
	\colorbox{grey}{
		\parbox[t]{1.1\textwidth}{ % Outer full width box
			\parbox[t]{1.02\textwidth}{ % Inner box for inner right text margin
				\raggedleft % Right align the text
				\fontsize{34pt}{40pt}\selectfont % Title font size, the first argument is the font size and the second is the line spacing, adjust depending on title length
				\vspace{0.7cm} % Space between the start of the title and the top of the grey box
				
				< Journey Assistant >\\
                Glossary\\
                Version 1.1\\
				
				\vspace{0.7cm} % Space between the end of the title and the bottom of the grey box
			}
		}
	}
	
	\vfill % Space between the title box and author information
	
	%------------------------------------------------
	%	Author name and information
	%------------------------------------------------
	
	\parbox[t]{1\textwidth}{ % Box to inset this section slightly
		\raggedleft % Right align the text
		\large % Increase the font size
		{\Large Group Member}\\[4pt] % Extra space after name
        Yiwen Song\\
        Zhihui Xie\\
        Weizhe Wang\\
        Huangfei Jiang\\
        Haoping Chen\\
		% Institution Name\\[4pt] % Extra space before URL
		% \texttt{LaTeXTemplates.com}\\
		
		\hfill\rule{0.2\linewidth}{1pt}% Horizontal line, first argument width, second thickness
    }
    
	
\end{titlepage}

\newpage

\begin{center}
    {\LARGE Modification History}
    
    \begin{tabular}{|c|c|c|c|} 
        \hline 
        Date&Version&Description&Author\\
        \hline  
        2019-04-02&1.0&The first version of this document.&Zhihui Xie\\
		\hline 
		2019-04-20&1.1&Add some new terms.&Zhihui Xie\\
		\hline
		& & & \\
		\hline
		& & & \\
		\hline
    \end{tabular}    
\end{center}

\newpage

\tableofcontents
\newpage

\section{Intruduction}
\subsection{Purpose}
The glossary aims to explain the terms used in our <Software Requirement Specification> document. We will specialize the meaning in order to both avoid any misunderstanding and provide reference to our developers.

\subsection{Scope}
The target system is our proposed system: Journey Assistant System.

\subsection{Bibliography}
\begin{itemize}
    \item[(1)] <Object Oriented Software Engineering (Version 3)> (Tsinghua University Press)
    \item[(2)] <IEEE Recommended Practice for Software Requirement Specifications> (IEEE Std 830-1998)
\end{itemize}

\subsection{Brief Description}
The glossary includes the definition in the documents of software system in alphabetical order.

\section{Definition}
\paragraph{\underline{Android}} Android is a mobile operating system developed by Google.

\paragraph{\underline{APP}} Application software (app for short) is software designed to perform a group of coordinated functions, tasks, or activities for the benefit of the user. 

\paragraph{\underline{Functional}} Some requirement that need to be realized.

\paragraph{\underline{GAN}} A generative adversarial network (GAN) is a class of machine learning systems. Two neural networks contest with each other in a zero-sum game framework. This technique can generate photographs that look at least superficially authentic to human observers, having many realistic characteristics. It is a form of unsupervised learning.

\paragraph{\underline{IM}} Instant messaging (IM) technology is a type of online chat that offers real-time text transmission over the Internet.

\paragraph{\underline{JAS}}  Our proposed system: Journey Assistant System.

\paragraph{\underline{Non-functional}} Some requirement that can not berealized but is indispensable to our system.

\paragraph{\underline{OS}} An operating system (OS) is system software that manages computer hardware and software resources and provides common services for computer programs.

\paragraph{\underline{TCP/IP}} The Internet protocol suite is the conceptual model and set of communications protocols used in the Internet and similar computer networks. It is commonly known as TCP/IP because the foundational protocols in the suite are the Transmission Control Protocol (TCP) and the Internet Protocol (IP).

\paragraph{\underline{UI}} The user interface (UI), in the industrial design field of human–computer interaction, is the space where interactions between humans and machines occur. The goal of this interaction is to allow effective operation and control of the machine from the human end, whilst the machine simultaneously feeds back information that aids the operators' decision-making process.

%----------------------------------------------------------------------------------------

\end{document}

